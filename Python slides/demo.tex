% Full instructions available at:
% https://github.com/elauksap/focus-beamertheme

\documentclass{beamer}
\usepackage{siunitx}
\usepackage{multicol}
\usepackage{tikz}
\usepackage[most]{tcolorbox}
\usepackage[utf8]{inputenc} % Para permitir acentos y ñ
\usetheme{focus}
\newtcolorbox[auto counter, number within=section]{Programexample}[2][]
{colback=green!5!white,colframe=magenta!75!black,
	fonttitle=\bfseries, title=Ejemplo: #2,#1}

\title{Introducción a Python}
\author{Paula Muñoz Lago \\ Esther Ruiz-Capillas Muñoz}
\titlegraphic{\includegraphics[scale=0.08]{python.png}}
\institute{Facultad de Informática, UCM}

\begin{document}
    \begin{frame}
        \maketitle
    \end{frame}

    \begin{frame} {Introducción a la programación}
	    \textbf{Programar: } Dar una serie de instrucciones a nuestro ordenador para que las cumpla con éxito y así obtener el resultado deseado.
	    
	    Los pasos que hay que seguir son los siguientes: 
	    \begin{enumerate}
	    	\item Escribir el código
	    	\item Compilarlo
	    	\item Ejecutarlo
	    \end{enumerate}
	\end{frame}

	\begin{frame} {Escribir el código}
		Dependiendo del sistema operativo que uses, y la cantidad de código que vayas a escribir, te recomiendamos algunos programas...
		
		\begin{itemize}
			\item \textbf{Notepad++ o SublimeText:} Son editores de texto que no disponen de compilador integrado, por lo que todo lo que programemos en ellos habrá que compilarlo y ejecutarlo desde la terminal de comandos. El primero se encuentra en Windows y el segundo en Linux.
			\item \textbf{Visual Studio Code: } También es un editor de texto, en el cual podremos escribir nuestro código e instalar a continuación extensiones (plug-ins) para poder ejecutar código. Se encuentra en ambos sistemas operativos.
		\end{itemize}
	\end{frame}

	\begin{frame}{Qué significa "Compilar"}
	\setlength{\parskip}{8mm}
	El proceso de compilación se encarga de comprobar que no hay fallos en el código escrito por el/la programador(a), y convertirlo a lenguaje de máquina (no legible por los programadores).
	
	Si la compilación da algún error, el proceso terminará y nos indicará la linea de código donde está el error.
	
	Para comprobar que la compilación ha terminado con éxito bastará con ver que se ha generado correctamente el archivo .exe (ejecutable)
	\end{frame}

	\begin{frame}{Ejemplo de compilación y ejecución}
	\begin{enumerate}
		\item Creamos Test.py y escribimos: 
		\begin{gather}
		\color{magenta}\textit{print("ASAP mola!")}
		\end{gather}
		\item Si trabajamos en Visual Studio Code y hemos instalado la extensión Code Runner bastará con darle al botón de "play" que se encuentra en la esquina superior derecha, si trabajamos desde un editor sin compilador, o no tenemos dicha extensión, bastará con que escribamos en la terminal el siguiente comando:
		\begin{gather}
			\color{magenta}\textit{python -u Test.py}
		\end{gather}
	
		De esta forma, veremos que en la pantalla aparece "ASAP mola!"
	\end{enumerate}
	\end{frame}
\section{Python}
    \begin{frame}{Razones por las que aprender Python}
    	\setlength{\parskip}{8mm} %separación entre parrafos
        Lenguaje "sencillo", ideal para iniciarse en la programación.
        
        Ligero, de compilación rápida.
        
        Además, es uno de los lenguajes pricipales con los que se trabaja en el campo del \textbf{procesamiento del lenguaje natural}\footnote{\url{https://es.wikipedia.org/wiki/Procesamiento_de_lenguajes_naturales}}
    \end{frame}
    \begin{frame}{Tipos de Datos}
    	En programación, siempre tenemos que guardar datos en algún punto de nuestro programa. Para ello, debemos conocer bien los diferentes tipos de datos con los que nos vamos a encontrar. En esta introducción veremos los tipos más comunes:
		\vspace{1cm}
		
	 \centering\hyperlink{datos:numericos}{\textbf{Numericos}}
		\hspace{1cm} \hyperlink{datos:strings}{\textbf{Strings}}
		\hspace{1cm} \hyperlink{datos:bool}{\textbf{Bool}}
		\hspace{1cm} \hyperlink{datos:listas}{\textbf{Listas}}
    \end{frame}
    
    \begin{frame} {Tipos de Datos: Numéricos}
    \label{datos:numericos}
		\begin{block}{Datos numéricos}
			int
			
			floating point
		\end{block}
		\begin{Programexample}{Prueba con datos de tipo int}
			\centering
			\textit{x = 2}
			
			\textit{y = 3}
			
			print(x + y)
		\end{Programexample}
	\begin{Programexample}{Prueba con datos de tipo float}
		\centering\textit{x = 2.56}
		
		 print(type(x))
	\end{Programexample}
		
    \end{frame}
    
    \begin{frame} {Tipos de Datos: Strings}
    \label{datos:strings}
    Un string es una cadena de caracteres, sobre las cuales pordemos operar de la siguiente forma:
    \begin{Programexample} {Operaciones sobre una cadena}
    	frase = \textcolor{orange}{"ASAP mola"}
    	
    	print(frase) \hspace{3cm} \textcolor{green}{"ASAP mola"}
    	
    	print(frase[3])\hspace{2.7cm} \textcolor{green}{"P"}
	\end{Programexample}

	Entenderemos las frases como una lista de caracteres, por eso en el segundo ejemplo, intentamos acceder a la posición 3 de la cadena  \textcolor{orange}{"ASAP mola"}.
	\textbf{En programación las listas empiezan en la posición 0}.
    \end{frame}
    \begin{frame}{Tipos de datos: Booleanos}
    %añadir más info
    \label{datos:bool}
    Estos tipos de datos únicamente son capaces de indicar si una variable es verdadera o falsa, se utilizarán más adelante como condición de control.
    
    \begin{Programexample}
    	\centering ASAPExiste = true
    \end{Programexample}

	Necesario conocer la \hyperlink{moorelaw}{Ley de Moore} para trabajar con Booleanos correctamente.
	\end{frame}
    \begin{frame}{Tipos de datos: Listas}
    \label{datos:listas}
    Al igual que en el último ejemplo, podemos crear listas de números, strings, o distintos tipos.
    
    \begin{Programexample}{Trabajo con listas}
    	info\_ASAP = \textcolor{orange}{["ASAP", "UCM", "Filología", 2018]}
    	\newline\newline
    	print("Información sobre la asociación",  info\_ASAP[0])\newline
    	print("Pertenece a la facultad: ", info\_ASAP[2], info\_ASAP[1]) \newline
    	print("Año de creacción: ", info\_ASAP[3])\newline
    	print("Numero de datos que disponemos de ", info\_ASAP[0], ":", len(info\_ASAP))
    \end{Programexample}
    \end{frame}

    \section{Estructuras}
    \begin{frame}{Leyes de Moore}
    \label{moorelaw}
       Antes de comenzar con nuestros primeros programas, debemos conocer esta ley:
       
    \end{frame}
    \begin{frame}{Estructuras de control: If}
	Todo programa informático necesita tomar decisiones, por lo que gracias a las condiciones Booleanas y a los bloques "if", dotaremos a nuestro programa de la posibilidad de ejecutar un fragmento de código u otro.
	%ver porqué no funciona el espaciado horizontal
	%explicar and or not antes de esto...
	\begin{Programexample} {Ejemplo del uso de if}
		socios = 300 \newline
		\textcolor{blue}{if} socios > 500:\newline
		\hspace{3cm} print(\textcolor{orange}{"ASAP tiene más de 500 socios, impresionante"})\newline
		\textcolor{blue}{elif} socios > 250 \textcolor{blue}{and} socios <= 500:\newline
		\hspace{2cm} print(\textcolor{orange}{"ASAP tiene entre 251 y 500 socios!"})\newline
		\textcolor{blue}{else}:\newline
		\hspace{2cm} print(\textcolor{orange}{"ASAP tiene "},socios, \textcolor{orange}{"socios"})
	\end{Programexample}
	\end{frame}
	\begin{frame}{Bucles: While}
	Para ejecutar un código varias veces, y hacerlo de una forma limpia y ordenada, siempre vamos a tener que hacer uso de algún tipo de bucle.
	\begin{Programexample}{Ejemplo erroneo}
		print(\textcolor{orange}{"ASAP mola"}) \newline
		print(\textcolor{orange}{"ASAP mola"})
	\end{Programexample}
	\begin{Programexample}{Ejemplo uso de bucle while}
		veces\_que\_lo\_he\_dicho = 0 \newline
		veces\_que\_quiero\_decirlo = 2 \newline
		\textcolor{blue}{while} veces\_que\_lo\_he\_dicho < veces\_que\_quiero\_decirlo:
		\hspace{2cm} print(\textcolor{orange}{"ASAP mola"}) \newline
		\hspace{2cm} veces\_que\_lo\_he\_dicho++
		
	\end{Programexample}
\end{frame}
\begin{frame}{Bucles: For}
Otro tipo de bucles comunmente utilizados en programación son los bucles for, en este caso encontramos diferentes sintaxis que pueden cumplir la misma funcionalidad.
\begin{Programexample}{Ejemplo uso de bucle for}
	veces\_que\_quiero\_decirlo = 2 \newline
	\textcolor{blue}{for} veces\_que\_lo\_he\_dicho \textcolor{blue}{in range}(veces\_que\_quiero\_decirlo): \newline
	\hspace{2cm} print(\textcolor{orange}{"ASAP mola"})
\end{Programexample}
\begin{Programexample}{Ejemplo uso de bucle for con listas}
	frutas = [\textcolor{orange}{"manzana", "platano", "naranja"}] \newline
	\textcolor{blue}{for} fruta \textcolor{blue}{in} frutas: \newline
	\textcolor{blue}{if} fruta \textcolor{blue}{is} "platano": \newline
	\hspace{2cm} print(\textcolor{orange}{"Gracias!"})
\end{Programexample}
\end{frame}
\section{Ahorcado}
\begin{frame}{Enunciado}
content...
\end{frame}

    \begin{frame}[focus]
        Muchas gracias por la atención! \newline
        Esperamos que esta charla haya despertado vuestro interés por Python y que sigais trabajando en ello!!
    \end{frame}

	\begin{frame}[focus]
	Paula Muñoz Lago pmunoz06@ucm.es \newline
	Esther Ruiz-Capillas Muñoz 
	\end{frame}

\end{document}
