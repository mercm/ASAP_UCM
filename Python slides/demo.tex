% Full instructions available at:
% https://github.com/elauksap/focus-beamertheme

\documentclass{beamer}
\usepackage[utf8]{inputenc} % Para permitir acentos y ñ
\usetheme{focus}

\title{Breve introducción a Python}
\author{Paula Muñoz Lago \\ Esther Ruiz-Capillas Muñoz}
\titlegraphic{\includegraphics[scale=0.08]{python.png}}
\institute{Facultad de Informática, UCM}

\begin{document}
    \begin{frame}
        \maketitle
    \end{frame}

    \begin{frame} {Introducción a la programación}
	    \textbf{Programar: } Dar una serie de instrucciones a nuestro ordenador para que las cumpla con éxito y así obtener el resultado deseado.
	    
	    Los pasos que hay que seguir son los siguientes: 
	    \begin{enumerate}
	    	\item Escribir el código
	    	\item Compilarlo
	    	\item Ejecutarlo
	    \end{enumerate}
	\end{frame}

	\begin{frame} {Escribir el código}
		Dependiendo del sistema operativo que uses, y la cantidad de código que vayas a escribir, te recomiendamos algunos programas...
		
		\begin{itemize}
			\item \textbf{Notepad++ o SublimeText:} Son editores de texto que no disponen de compilador integrado, por lo que todo lo que programemos en ellos habrá que compilarlo y ejecutarlo desde la terminal de comandos. El primero se encuentra en Windows y el segundo en Linux.
			\item \textbf{Visual Studio Code: } También es un editor de texto, en el cual podremos escribir nuestro código e instalar a continuación extensiones (plug-ins) para poder ejecutar código. Se encuentra en ambos sistemas operativos.
		\end{itemize}
	\end{frame}

	\begin{frame}{Qué significa "Compilar"}
	\setlength{\parskip}{8mm}
	El proceso de compilación se encarga de comprobar que no hay fallos en el código escrito por el/la programador(a), y convertirlo a lenguaje de máquina (no legible por los programadores).
	
	Si la compilación da algún error, el proceso terminará y nos indicará la linea de código donde está el error.
	
	Para comprobar que la compilación ha terminado con éxito bastará con ver que se ha generado correctamente el archivo .exe (ejecutable)
	\end{frame}

	\begin{frame}{Ejemplo de compilación y ejecución}
	\begin{enumerate}
		\item Creamos Test.py y escribimos: 
		\begin{gather}
		\textit{print("ASAP mola!")}
		\end{gather}
		\item Si trabajamos en Visual Studio Code y hemos instalado la extensión Code Runner bastará con darle al botón de "play" que se encuentra en la esquina superior derecha, si trabajamos desde un editor sin compilador, o no tenemos dicha extensión, bastará con que escribamos en la terminal el siguiente comando:
		\begin{gather}
			\textit{python -u Test.py}
		\end{gather}
	
		De esta forma, veremos que en la pantalla aparece "ASAP mola!"
	\end{enumerate}
	\end{frame}
    \begin{frame}{Razones por las que aprender Python}
    	\setlength{\parskip}{8mm} %separación entre parrafos
        Lenguaje "sencillo", ideal para iniciarse en la programación.
        
        Ligero, de compilación rápida (A continuación revisaremos éste concepto).
        
        Además, se utiliza mucho en campos como el \textbf{procesamiento del lenguaje natural}\footnote{\url{https://es.wikipedia.org/wiki/Procesamiento_de_lenguajes_naturales}}
    \end{frame}

    \begin{frame}
        This is a frame with plain style and it is numbered.
    \end{frame}
    
    \begin{frame}[t]
        This frame has an empty title and is aligned to top.
    \end{frame}
    
    \begin{frame}[noframenumbering]{No frame numbering}
        This frame is not numbered and is citing reference \cite{knuth74}.
    \end{frame}
    
    \begin{frame}{Typesetting and Math}
        The packages \texttt{inputenc} and \texttt{FiraSans}\footnote{\url{https://fonts.google.com/specimen/Fira+Sans}}\textsuperscript{,}\footnote{\url{http://mozilla.github.io/Fira/}} are used to properly set the main fonts.
        \vfill
        This theme provides styling commands to typeset \emph{emphasized}, \alert{alerted}, \textbf{bold}, \textcolor{example}{example text}, \dots
        \vfill
        \texttt{FiraSans} also provides support for mathematical symbols:
        \begin{equation*}
            e^{i\pi} + 1 = 0.
        \end{equation*}
    \end{frame}

    \section{Section 2}
    \begin{frame}{Blocks}
        \begin{block}{Block}
            Text.
        \end{block}
        \pause
        \begin{alertblock}{Alert block}
            Alert \alert{text}.
        \end{alertblock}
        \pause
        \begin{exampleblock}{Example block}
            Example \textcolor{example}{text}.
        \end{exampleblock}
    \end{frame}
    
    \begin{frame}{Lists}
        \begin{columns}[t, onlytextwidth]
            \column{0.33\textwidth}
                Items:
                \begin{itemize}
                    \item Item 1
                    \begin{itemize}
                        \item Subitem 1.1
                        \item Subitem 1.2
                    \end{itemize}
                    \item Item 2
                    \item Item 3
                \end{itemize}
            
            \column{0.33\textwidth}
                Enumerations:
                \begin{enumerate}
                    \item First
                    \item Second
                    \begin{enumerate}
                        \item Sub-first
                        \item Sub-second
                    \end{enumerate}
                    \item Third
                \end{enumerate}
            
            \column{0.33\textwidth}
                Descriptions:
                \begin{description}
                    \item[First] Yes.
                    \item[Second] No.
                \end{description}
        \end{columns}
    \end{frame}

    \begin{frame}[focus]
        Thanks for using \textbf{Focus}!
    \end{frame}
    
    \appendix
    \begin{frame}{References}
        \nocite{*}
        \bibliography{demo_bibliography}
        \bibliographystyle{plain}
    \end{frame}
    
    \begin{frame}{Backup frame}
        \usebeamercolor[fg]{normal text}
        This is a backup frame, useful to include additional material for questions from the audience.
        \vfill
        The package \texttt{appendixnumberbeamer} is used not to number appendix frames.
    \end{frame}
\end{document}
